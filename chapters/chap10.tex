\section{Notation}
\begin{itemize}
    \item $m = $ The number of training examples
    \item $x = $ The input variables / features
    \item $y = $ The output variable/"target" variables
    \item $(x_i, y_i) = $ A specific example ($i_{th}$ training example)
\end{itemize}

Passing the features into a learning algorithm that outputs a function (denoted $h = $ hypothesis), resulting in the function:
\[
    h_\theta(x) = \theta_0+\theta_1 x
\]
where $\theta_i$ are parameters, $\theta_0$ is the zero condition, and $\theta_1$ is the gradient.

\section{Cost Function}
A cost function lets us figure out how to fit the best straight line to our data.
We want to solve a minimization problem where we minimize $(h_\theta(x)-y)^2$.

The most common cost function is the mean squared error (MSE):
\[
    J(\theta_0, \theta_1) = \frac{1}{m}\sum_{i=1}^m\left(h_\theta(x_i)-y_i\right)^2
\]

Note that multiplying the cost function by $\frac{1}{2}$ does not change the location of the minima, which is why we use it to form the following cost function:
\bigskip

One half mean squared error:
\[
    J(\theta_0, \theta_1) = \frac{1}{2m}\sum_{i=1}^m\left(h_\theta(x_i)-y_i\right)^2
\]

The cost function determines the parameters, and the values associated with the parameters determine how the hypothesis behaves.
Different values generate different hypotheses.

The MSE cost function is convex.

Finding the global minima in linear regression can be done with a gradient descent algorithm.

\subsection{Cost Function for Multiple Variables}
\[
    J(\theta_0, \theta_1, \cdots, \theta_n) = \frac{1}{2m}\sum_{i=1}^m\left(h_\theta(x_i)-y_i\right)^2
\]

\section{Batch Gradient Descent Algorithm (BGD)}
\begin{itemize}
    \item In each step you look at all the training data
\end{itemize}

\subsubsection{Progress}
\begin{enumerate}
    \item Start with initial guesses, i.e. $(0, 0)$
    \item Keep changing $\theta_0$ and $\theta_1$ a little bit to reduct $J(\theta_0, \theta_1)$
    \item Each time the parameters change, select the gradient which reduces $J(\theta_0, \theta_1)$ the most
    \item Repeat until we converge to a local/global minima
\end{enumerate}
Different starting points will often result in different minimas.

\subsubsection{Formal progress}
\begin{itemize}
    \item Do the following until convergence:
    \begin{itemize}
        \item $\theta_j:=\theta_j-\alpha\frac{\partial}{\partial\theta_j}J(\theta_0, \theta_1)$
        \item For $j=0$ and $j=1$ in this example.
    \end{itemize}
    \item $\alpha$ is the learning rate
    \item $\alpha$ controls how aggressive the gradient descent is
\end{itemize}

"For $j=0$ and $j=1$" means that we update both parameters simultaneously. This is done by first solving the derivative for $\theta_0$ and storing it in a temporary variable. Do the same for $\theta_1$, and then update the original $\theta_j$-values.

\subsubsection{Partial Derivatives}
\[
    \frac{\partial}{\partial \theta_j}J(\theta_0, \theta_1) = \frac{\partial}{\partial\theta_j}\frac{1}{2m}\sum_{i=1}^m(\theta_0+\theta_1x_i-y_i)^2
\]
\[
    \frac{\partial}{\partial \theta_0}J(\theta_0, \theta_1) = \frac{1}{m}\sum_{i=1}^m(\theta_0+\theta_1x_i-y_i)*1
\]
\[
    \frac{\partial}{\partial \theta_1}J(\theta_0, \theta_1) = \frac{1}{m}\sum_{i=1}^m(\theta_0+\theta_1x_i-y_i)x_i^{(1)}
\]
\[
    \frac{\partial}{\partial \theta_j}J(\theta_0, \cdots, \theta_n) = \frac{1}{m}\sum_{i=1}^m(\theta_0+\cdots+\theta_nx_i-y_i)x_i^{(j)}
\]

\subsection{Gradient Descent in Practice}
\begin{itemize}
    \item Feature Scaling (normalize)
    \begin{itemize}
        \item Problems with multiple features should be scaled to a more even scale across the features.
        \item One common way is to use the mean to scale all data into the range $\left[-1, 1\right]$.
        \item Try to create a range between $\left[-3, 3\right]$ and $\left[-\frac{1}{3}, \frac{1}{3}\right]$.
        \item This is done by $x_i = \frac{x_i-\hat{x}}{max(x)-min(x)}$
    \end{itemize}
    \item Running gradient descent on this kind of cost function can take a long time to find the global minima.
\end{itemize}

\section{Practical Tips}
\begin{itemize}
    \item Plot $min(J(\theta))$ with the number of iterations to clearly illustrate the convergence.
    \item Start with a low learning rate, and gradually increase it. It is normal to go for roughly treefold increments (multiply by 3).
    \item We could also use a polynomial regression on the form $(\theta_0 + \theta_1x + \theta_2x^2)$.
\end{itemize}
