\section{Summary Statistics}
\begin{itemize}
    \item Frequency
    \item Mode
    \item Percentiles
    \item Mean
    \item Median
    \item Range
    \item Variance
\end{itemize}

\begin{theo}[Mean]{theo:theo5}
    \label{eq:mean}
        \[
            mean(x) = \bar{x} = \frac{1}{m}\sum_{i=1}^m x_i
        \]
\end{theo}

\begin{theo}[Median]{theo:theo6}
    \label{eq:median}
        \[
            median(x) = 
            \left\{\begin{matrix}
                x_{(r+1)} & \text{if } m=2r+1 \\ 
                \frac{1}{2} \left( x_{(r)} + x_{(r+1)} \right)  & \text{if } m=2r
            \end{matrix}\right.
        \]
\end{theo}

\begin{theo}[Variance]{theo:theo7}
    \label{eq:variance}
        \[
            variance(x) = s_x^2 = \frac{1}{m-1} \sum_{i=1}^m(x_i-\bar{x})^2
        \]
\end{theo}

\begin{theo}[Average Absolute Deviation]{theo:theo8}
    \label{eq:aad}
        \[
            AAD(x) = \frac{1}{m} \sum_{i=1}^m \abs{x_i-\bar{x}}
        \]
\end{theo}

\begin{theo}[Mean Absolute Deviation]{theo:theo9}
    \label{eq:mad}
        \[
            MAD(x) = median \left( \left\{\abs{x_1-\bar{x}} \cdots \abs{x_m-\bar{x}}\right\} \right)
        \]
\end{theo}

\section{Visualisation}
Visualisation is the conversion of data into a visual or tabular format so that the characteristics of the data and the relationships among data items or attributes can be analyzed or reported.
Visualisation allows humans to detect general patterns and trends, as well as detect outliers and unusual patterns.

\subsection{Representation}
\subsection{Arrangement}
\subsection{Selection}

\subsection{Visualisation Techniques}





