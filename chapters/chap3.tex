\section{Summary Statistics}
\begin{itemize}
    \item Frequency
    \item Mode
    \item Percentiles
    \item Mean
    \item Median
    \item Range
    \item Variance
\end{itemize}

\begin{theo}[Mean]{theo:theo5}
    \label{eq:mean}
        \[
            mean(x) = \bar{x} = \frac{1}{m}\sum_{i=1}^m x_i
        \]
\end{theo}

\begin{theo}[Median]{theo:theo6}
    \label{eq:median}
        \[
            median(x) = 
            \left\{\begin{matrix}
                x_{(r+1)} & \text{if } m=2r+1 \\ 
                \frac{1}{2} \left( x_{(r)} + x_{(r+1)} \right)  & \text{if } m=2r
            \end{matrix}\right.
        \]
\end{theo}

\begin{theo}[Variance]{theo:theo7}
    \label{eq:variance}
        \[
            variance(x) = s_x^2 = \frac{1}{m-1} \sum_{i=1}^m(x_i-\bar{x})^2
        \]
\end{theo}

\begin{theo}[Average Absolute Deviation]{theo:theo8}
    \label{eq:aad}
        \[
            AAD(x) = \frac{1}{m} \sum_{i=1}^m \abs{x_i-\bar{x}}
        \]
\end{theo}

\begin{theo}[Mean Absolute Deviation]{theo:theo9}
    \label{eq:mad}
        \[
            MAD(x) = median \left( \left\{\abs{x_1-\bar{x}} \cdots \abs{x_m-\bar{x}}\right\} \right)
        \]
\end{theo}

\section{Visualisation}
Visualisation is the conversion of data into a visual or tabular format so that the characteristics of the data and the relationships among data items or attributes can be analyzed or reported.
Visualisation allows humans to detect general patterns and trends, as well as detect outliers and unusual patterns.

\subsection{Representation}
The representation is the mapping of information to a visual format. Data objects, their attributes, and the relationships among data objects are translated into graphichal elements such as points, lines, shapes, and colors.

\subsection{Arrangement}
Arrangement is the placement of visual elements within a display. This can make a large difference in how easy it is to understand the data.

\subsection{Selection}
Selection is the elimination or the de-emphasis of certain objects and attributes.
Such selections may involve choosing a subset of attributes, or choosing a subset of objects. Only selecting some attributes can be done through dimensionality reduction, while selecting some objects can be done through stratified sampling.
Some things to keep in mind when using data selection is to contain the diversity of the objects.

\subsection{Visualisation Techniques}
\begin{itemize}
    \item Histogram plot
    \item Box plot
    \item Scatter plot
    \item Matrix plot
    \item Parallel coordinates plot
    \item Star plot
\end{itemize}
